\documentclass[Main]{subfiles}

\begin{document}

\chapter{Unit test}
Beskriv, hvorfor der skal laves Unit test for jeres system.
Eksempel:

\section{Applikation}
Herunder ses nogle af de unit tests der er kørt for applikationen\dots

\subsection{Modtag tæller}
Beskrivelse af hvad der skal gøres for, at køre en test der viser, at der kan gemmes filer. 
Disse køres, hvis muligt, kun med kode -- selve applikationen skal derfor ikke åbnes.

\begin{itemize}
\item Åben C\#-projekt \textit{The App}.
\item Åben \code{TheApp.Test.Unit} $\rightarrow$ \code{Counter.cs}
\item Start unit test ved tryk på \code{Reshaper} $\rightarrow$ \code{Unit Tests} $\rightarrow$ \code{Run Unit Test}
\item Testresultatet vil vises i et vindue med grønt, såfremt den er succesfuld.
\end{itemize}

Den testede kode er vist i \codeTitle \ref{lst:Get}

\begin{lstlisting}[caption=Returnering af counter, style=Code-C++, label=lst:Get]
using NUnit.Framework;

namespace Test
{
	public class UnitTest
	{
		private ICounter _counter;

		[SetUp]
		public void TestInit()
		{	
			_counter = new Counter();
		}

		[Test]
		public void GetCounter()
		{
			var value = _counter.GetCounter();
			Assert.AreEqual(0, value);
		}
		
		...
	}
}
\end{lstlisting}



\subsection{Sæt tæller}
For at sætte tælleren skrives først til den, hvorefter den aflæses.
\codeTitle \ref{lst:Set} viser test af Setter-funktionen.
\begin{lstlisting}[caption=Setter-funktion, style=Code-C, label=lst:Set]
	...
	[Test]
	public void SetCounterTest()
	{
		_counter.SetCounter(2);
		var value = _counter.GetCounter();
		Assert.AreEqual(2, value);
	}
	...
\end{lstlisting}


\end{document}