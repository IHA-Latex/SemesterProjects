\documentclass[Main]{subfiles}

\begin{document}

\chapter{Integrationstests}

Beskriv hvorfor der udføres integrationstest.
Eksempel:

\section{Applikationen}
For at sikre, at der bliver foretaget specifikke kald udføres en integrationstest, \codeTitle \ref{lst:int1}.

\begin{lstlisting}[caption=Integrationstest, style=Code-Java, label=lst:int1]
	public class IntegrationTest
	{
		private Mock<ICounter> _counterMock;
		private IColletor _colletor;

		[SetUp]
		public void Setup()
		{
			_counterMock = new Mock<ICounter>();
			_colletor = new Collector(_counterMock.Object);
		}

		[TearDown]
		public void TearDown()
		{
			_counterMock = null;
			_colletor = null;
		}

		[Test]
		public void CheckUnderTwo()
		{
			_counterMock.Setup(p => p.GetCounter()).Returns(1);
			_colletor.Check();
			
			//Is SetStatus called?
			_counterMock.Verify(p => p.SetStatus(true), Times.Never);
		}

		[Test]
		public void CheckOverTwo()
		{
			_counterMock.Setup(p => p.GetCounter()).Returns(3);
			_colletor.Check();
			
			//Is SetStatus called?
			_counterMock.Verify(p => p.SetStatus(true), Times.AtLeastOnce);
		}
	}
\end{lstlisting}

Først oprettes der en mock at \code{ICounter}, som får én bestemt værdi, der checkes af en  klasse af typen \code{ICollector}. 
Såfremt den pågældende værdi er under 2 skal \code{IColletor}'s funktion ikke kaldes.
Hvis den er skal den kaldes én gang.






\section{Aflæsning på µ-controller}
Test af kommunikation fra µ-controller til temperaturmåler.

\subsection{Læsning af temperatur}
For at læse temperaturen bruges bestemt udstyr samt bestemte forbindelser.

\paragraph{Udstyr}

\begin{itemize}
\item STK500 kit
\item LM75 board
\end{itemize}


\paragraph{Forbindelser:}

\begin{table}[H]
\centering
\begin{tabular}{c c c c c}
\hline 
STK500 & LM75 &  STK500 \\ 
\hline 
PB0 & $\bullet$ & RXD \\
PB1 & $\bullet$ & TXD \\ 
PC0 & SCL & $\bullet$ \\
PC1 & SDA & $\bullet$ \\
PC8 og PC9 & GND og VCC & $\bullet$ \\
\hline 
\end{tabular} 
\caption{Hardware forbindelser}
\label{tbl:oversigt}
\end{table}

\codeTitle \ref{lst:LM75Read} viser det kode der skal til for, at aflæse temperaturmåleren.

\begin{lstlisting}[caption=Funktion til læsning af LM75, style=Code-C, label=lst:LM75Read]
int LM75_temperature(unsigned char SensorAddress)
{
	i2c_start();
	
	unsigned char address = ((0b01001000 | SensorAddress) << 1) | 0x01;
	i2c_write(address); //Address write
	
	unsigned char tempMSB = i2c_read(0x00); 
	
	unsigned char tempLSB = i2c_read(0x01);
	
	i2c_stop();
	
	return (tempLSB>>7) | (tempMSB<<1);
}

\end{lstlisting}

For at teste \codeTitle \ref{lst:LM75Read} køres \codeTitle \ref{lst:LM75Reading}.

\begin{lstlisting}[caption=Funktion til læsning af LM75, style=Code-C, label=lst:LM75Reading]
#include <avr/delay.h>
#include "UART.h"

void main()
{
	int reading = LM75_temperature(0);
	SendInteger(reading);
	_delay_ms(250);
}
\end{lstlisting}






\end{document}