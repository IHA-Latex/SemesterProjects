\documentclass[Main]{subfiles}

\begin{document}

\chapter{Integrationstests}

Når de enkelte enheder var testet blev større dele sammensat for at se, om de kunne spille sammen.
Ved at køre det store projekt kan forskellige integrationstests vises.

\subsection{Dronen}
I filen \code{InoHelp.cpp} ligger følgende funktion, der køres 5 gange i sekundet:

\begin{lstlisting}[caption=process2HzTask()'s integrationstest, style=Code-C, label=lst:ref]
void process2HzTask()
{
	//PrintSonarReport();
	//PrintChosenProgram();
	//PrintControllerOutput();
	//PrintWarnings();
}
\end{lstlisting}

\begin{itemize}
\item Ved udkommentering af \code{PrintSonarReport()} printes dronens aktuelle højde, om den er i 'free flight' (ikke fastlagt højde) eller 'Fixed flight' (fastlagt højde), hvilken højden den skal ramme (såfremt den er i 'Fixed flight'), samt venstre-, front- og højre sonarsensorens aktuelle målinger.

\item Ved udkommentering af \code{PrintChosenProgram()} udskrives det valgte program.

\item Ved udkommentering af \code{PrintControllerOutput()} udskrives de værdier der sendes til \code{Receiver}-klassen for x-, y- og z-aksen, throttle-værdien samt om den skal holde sin højde eller ej.

\item Ved udkommentering af \code{PrintWarnings()} advarsler for hver sonar, såfremt der er nogen.
\end{itemize}


\newpage
I filen \code{ControlFaker.cpp} ligger følgende funtion, der køres 50 gange i sekunded.


\begin{lstlisting}[caption=lst:PrintMotorOutput()'s integrationstest, style=Code-C, label=lst:PrintMotorOutput]
void PrintMotorOutput()
{
	if(!IsMotorKilled())
	{
		//printInLine("Motor output: ", MOTORMODE);
		//printInLine(motorCommand[0], MOTORMODE);
		//printInLine(", ", MOTORMODE);
		//printInLine(motorCommand[1], MOTORMODE);
		//printInLine(", ", MOTORMODE);
		//printInLine(motorCommand[2], MOTORMODE);
		//printInLine(", ", MOTORMODE);
		//printInLine(motorCommand[3], MOTORMODE);
		//printInLine(", ", MOTORMODE);
		//printInLine(_controllerInput[THROTTLE], MOTORMODE);
		//println(MOTORMODE);
	}
}
\end{lstlisting}

Såfremt disse linjer udkommenteres vil den værdi, der sendes til hver enkelt motor på dronen, samt den værdi de skulle holde, blive udskrevet.
Dette er bl.a. brugt til at generere den graf der er vist i design-dokumentet\cite{Design}  i afsnit 2.1.1.4.


\subsection{Kommunikation mellem sender Radio og modtager radio}
Tes om det er muligt at sende en besked fra sender til modtager

\textbf{Opsætning:}\\
\textbf{Udstyr:}
\begin{itemize}
\item 2 stk STK500 kit
\item 2 stk Radio fra projektet
\item 2 stk Adapter til stk500 kit
\end{itemize}


\textbf{Forbindelser:}
\begin{itemize}
\item PORTC -> Led port
\end{itemize}

\begin{itemize}
\item \textbf{Radio -> 	STK500}
\item CSN	->	PB2
\item GDO0	->	PD2
\item GND	->	GND
\item SO	->	PB4
\item SCK	->	PB5
\item Si	->	PB3
\item +3.3V	->	VTG
\end{itemize}


Vtarget på begge STK500 skal sættes til 3.3V ned fra de normale 5V.

I denne test sendes en pakke fra µ-kontroller 1 over radioen til radio og til µ-kontroller 2. Pakken vises så på led porten på STK500 kit nr. 2.

\begin{lstlisting}[caption=Radio integrationstest, style=Code-C, label=lst:PrintMotorOutput]

#include <avr/io.h>
#include <util/delay.h>
#include <avr/sleep.h>
#include <avr/interrupt.h>

#define F_CPU 3686400
#define CE_PIN 4
char data[TWI_BUFFER_SIZE];
unsigned char messageBuf[TWI_BUFFER_SIZE];
char i;

void SPI_MasterInit(void)         //Initialize Atmega8 as master
{
	
	DDRB = (1<<DDB5)|(1<<DDB3)|(1<<DDB2);   // Set MOSI,SCK and SS\ output, all others input
	
	SPCR = (1<<SPE)|(1<<MSTR)|(1<<SPR0);   //  Enable SPI, Master, set clock rate fck/16
}

void CSn(int i)               //Toggle CSn
{
	if(i==1)
	PORTB|=0x04;
	else
	PORTB&=0xFB;
}

unsigned char SPI_transmit(unsigned char data)
{
	// Start transmission
	SPDR = data;

	// Wait for transmission complete
	while(!(SPSR & (1<<SPIF)));
	data = SPDR;

	return(data);
}

void Full_RegConfig(void)         //for full register configuration using BURST access
{
	CSn(0);
	while(bit_is_set(PINB,4)){}
SPI_transmit(0x3E);      //address for patable access
SPI_transmit(0x60);      // --ss1--//

//selve opsetningen

SPI_transmit(0x40);      //-- Starter fra 00 med burst--//

//GDO Settings
SPI_transmit(0x02);      // register 0x00
SPI_transmit(0x2E);      // register 0x01
SPI_transmit(0x01);      // register 0x02
SPI_transmit(0x40);      // register 0x03

SPI_transmit(0xD3);      // register 0x04
SPI_transmit(0x91);      // register 0x05
SPI_transmit(0xFF);      // register 0x06
SPI_transmit(0x0C);      // register 0x07
SPI_transmit(0x0D);      // register 0x08
SPI_transmit(0x00);      // register 0x09
SPI_transmit(0x00);      // register 0x0A
SPI_transmit(0x06);      // register 0x0B
SPI_transmit(0x00);      // register 0x0C
SPI_transmit(0x10);      // register 0x0D
SPI_transmit(0xB1);      // register 0x0E
SPI_transmit(0x3B);      // register 0x0F
SPI_transmit(0xF6);      // register 0x10 
SPI_transmit(0x83);      // register 0x11 
SPI_transmit(0x13);      // register 0x12
SPI_transmit(0x22);      // register 0x13
SPI_transmit(0xF8);      // register 0x14
SPI_transmit(0x15);      // register 0x15
SPI_transmit(0x07);      // register 0x16
SPI_transmit(0x30);      // register 0x17
SPI_transmit(0x18);      // register 0x18

SPI_transmit(0x16);      // register 0x19
SPI_transmit(0x6C);      // register 0x1A
SPI_transmit(0x03);      // register 0x1B
SPI_transmit(0x40);      // register 0x1C
SPI_transmit(0x91);      // register 0x1D
SPI_transmit(0x87);      // register 0x1E
SPI_transmit(0x6B);      // register 0x1F
SPI_transmit(0xFB);      // register 0x20
SPI_transmit(0x56);      // register 0x21
SPI_transmit(0x10);      // register 0x22

SPI_transmit(0xE9);      // register 0x23
SPI_transmit(0x2A);      // register 0x24
SPI_transmit(0x00);      // register 0x25
SPI_transmit(0x1F);      // register 0x26

SPI_transmit(0x41);      // register 0x27
SPI_transmit(0x00);      // register 0x28
SPI_transmit(0x59);      // register 0x29
SPI_transmit(0x7F);      // register 0x2A
SPI_transmit(0x3F);      // register 0x2B
SPI_transmit(0x81);      // register 0x2C
SPI_transmit(0x35);      // register 0x2D
SPI_transmit(0x09);      // register 0x2E

CSn(1);

}

void initExtInts()
{
	// INT0:Rising edge
	MCUCR = 0b00000011;
	// Enable external interrupts INT0
	GICR |= 0b01000000;
}
// Interrupt service routine for INT0
ISR (INT0_vect)
{
	char info;	
	CSn(0);
	SPI_transmit(0xFB); //Number of bytes
	info = SPI_transmit(0x3D);//rand
	CSn(1);
	
	//read the data
	CSn(0);
	SPI_transmit(0xFF); //strobe - Exit all
	for (i=0;i<(info-1);i++)
	{
		data[i] = (SPI_transmit(0x00)>>1);
	}
	CSn(1);
	PORTC = ~data[0];
	
	CSn(0);
	SPI_transmit(0x36); //strobe - Exit all
	SPI_transmit(0x3A); //strobe - flush RX
	SPI_transmit(0x34); //enable Rx
	CSn(1);
}

int main()
{
	DDRC = 0b11111111;
	PORTC =0b00000000;

	// Initialize  INT0
	initExtInts();
	SPI_MasterInit();         //initialize SPI
				
	CSn(0);
	SPI_transmit(0x30);
	CSn(1);
	
	Full_RegConfig();
	
	CSn(0);
	SPI_transmit(0x36); //strobe - Exit all
	SPI_transmit(0x3A); //strobe - flush RX
	CSn(1);
		
	CSn(0);
	SPI_transmit(0x34); //enable Rx
	CSn(1);
	
	//Global interrupt enable
	sei();

	TWI_Start_Transceiver();

  while (1)
  {
  }
}
\end{lstlisting}
\end{document}