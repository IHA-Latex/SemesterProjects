\documentclass[Main]{subfiles}

\begin{document}

\chapter{Unit test}
Beskriv, hvorfor der skal laves Unit test for jeres system.

\section{Applikation}
Herunder ses nogle af de unit tests der er kørt for applikationen\dots

\subsection{Modtag tæller}
Beskrivelse af hvad der skal gøres for, at køre en test der viser, at der kan gemmes filer. 
Disse køres, hvis muligt, kun med kode -- selve applikationen skal derfor ikke åbnes.

\begin{itemize}
\item Åben C\#-projekt \textit{The App}.
\item Åben \code{TheApp.Test.Unit} $\rightarrow$ \code{Counter.cs}
\item Start unit test ved tryk på \code{Reshaper} $\rightarrow$ \code{Unit Tests} $\rightarrow$ \code{Run Unit Test}
\item Testresultatet vil vises i et vindue med grønt, såfremt den er succesfuld.
\end{itemize}

Den testede kode er vist i \codeTitle \ref{lst:Get}

\begin{lstlisting}[caption=Returnering af counter, style=Code-C++, label=lst:Get]
using NUnit.Framework;

namespace Test
{
	public class UnitTest
	{
		private ICounter _counter;

		[SetUp]
		public void TestInit()
		{	
			_counter = new Counter();
		}

		[Test]
		public void GetCounter()
		{
			var value = _counter.GetCounter();
			Assert.AreEqual(0, value);
		}
		
		...
	}
}
\end{lstlisting}



\subsection{Sæt tæller}
For at sætte tælleren skrives først til den, hvorefter den aflæses.

\codeTitle \ref{lst:Set} viser test af Setter-funktionen.
\begin{lstlisting}[caption=Setter-funktion, style=Code-C, label=lst:Set]
	...
	[Test]
	public void SetCounterTest()
	{
		_counter.SetCounter(2);
		var value = _counter.GetCounter();
		Assert.AreEqual(2, value);
	}
	...
\end{lstlisting}



\newpage


\newpage
\section{Aflæsning på µ-controller}
Test om µ-controller kunne læse temperaturmåler.

\paragraph{Udstyr}

\begin{itemize}
\item STK500 kit
\item LM75 board
\end{itemize}


\paragraph{Forbindelser:}

\begin{table}[ht]
\centering
\begin{tabular}{c c c c c}
\hline 
STK500 & LM75 &  STK500 \\ 
\hline 
PB0 & $\bullet$ & RXD \\
PB1 & $\bullet$ & TXD \\ 
PC0 & SCL & $\bullet$ \\
PC1 & SDA & $\bullet$ \\
PC8 og PC9 & GND og VCC & $\bullet$ \\
\hline 
\end{tabular} 
\caption{Hardware forbindelser}
\label{tbl:oversigt}
\end{table}


Vtarget på STK500 skal sættes til 3.3V ned fra de normale 5V.

Radioen har nogle test register der kan læses fra. "Test0", "Test1" og "Test2" 

I dette forsøg læsers der fra Test2.
"This register will be forced to 0x88 or 0x81 when it wakes up from
SLEEP mode, depending on the configuration of FIFOTHR." \cite{TI-cc1101}

Med testkoden, skal svaret være "0x88". Resultatet kan ses på led porten på STK500 kittet.

\begin{lstlisting}[caption=SPI test, style=Code-C, label=lst:itocLEs]
// includes
#include <avr/io.h>
#include <util/delay.h>
#include <avr/sleep.h>


// variables

void SPI_MasterInit(void)         //Initialize Atmega8 as master
{
	DDRB = (1<<DDB5)|(1<<DDB3)|(1<<DDB2);   // Set MOSI,SCK and SS\ output, all others input
	SPCR = (1<<SPE)|(1<<MSTR)|(1<<SPR0);   //  Enable SPI, Master, set clock rate fck/16
}

//Toggle CSn
void CSn(int i)
{
	if(i==1)
	PORTB|=0x04;
	else
	PORTB&=0xFB;
}


unsigned char SPI_transmit(unsigned char data)
{
	// Start transmission
	SPDR = data;

	// Wait for transmission complete
	while(!(SPSR & (1<<SPIF)));
	data = SPDR;

	return(data);
}

int main()
{
    DDRC = 0xFF;
	SPI_MasterInit();         //initialize SPI
	
	CSn(0);
	//This register will be forced to 0x88 or 0x81 when it wakes up from SLEEP mode
	SPI_transmit(0xAE); //Read test 2 register
	PORTC = SPI_transmit(0x3B); //strobe - flush TX
	CSn(1);

	while (1)
	{

	}
}
\end{lstlisting}





\end{document}